% pdflatex notes.tex && pdflatex notes.tex

\documentclass{article}
\usepackage{amsfonts}
\usepackage{pifont}
\usepackage{hyperref}
\usepackage{amsmath}
\usepackage{amssymb}
\usepackage[most]{tcolorbox}
\usepackage{empheq}
\usepackage{geometry}
\usepackage{amsthm}
\usepackage{multicol}
\geometry{a4paper, margin=1cm, includefoot}

\theoremstyle{plain}
\newtheorem{definition}{Definition}[section]

\newtcbox{\mymath}[1][]{%
    nobeforeafter, math upper, tcbox raise base,
    enhanced, colframe=blue!30!black,
    colback=blue!15, boxrule=1pt,
    #1}

\newcommand{\myeq}[2]{
    \begin{multicols}{2}
            #1
    \columnbreak
        \\
        \\
        #2
    \end{multicols}
    \hrulefill
}

\begin{document}

\title{Notes on \href{https://books-library.net/files/books-library.net-02190127Xu1L6.pdf}{Discrete Time Signal Processing 3}}
\author{Rom Parnichkun}

\maketitle

\section{Discrete Time Signals}

\hrulefill
\myeq{
    \begin{equation}
        x[n] = x(nT)
    \end{equation}
}{A discrete signal $x$ sampled at period $T$.}

\subsection{Basic sequences}

\myeq{
    \begin{equation}
        \delta[n] = \begin{cases}
            0 & n \neq 0 \\
            1 & n = 0
        \end{cases}
    \end{equation}
}{This is the \textbf{unit sample} (impulse) function.}

\myeq{
    \begin{equation}
        x[n] = \sum_{k = -\infty}^{\infty}{\delta[k-n]x[k]}
    \end{equation}
}{
    This derivation seems quite redundant, however it will be useful later on when analyzing linear time invariant systems.}

\myeq{
    \begin{equation}
        u[n] = \begin{cases}
            0 & n < 0 \\
            1 & n \geq 0
        \end{cases}
        = \sum_{k = -\infty}^{\infty}{\delta} = \sum_{k = 0}^{\infty}{\delta[n-k]}
    \end{equation}
}{This is the \textbf{unit step} function.}

\myeq{
    \begin{equation}
        x[n] = A\alpha^n \label{eq:general_exponential}
    \end{equation}
}{General exponential sequence.}

\myeq{
    \begin{align}
        x[n] &= |A||\alpha|^n e^{(\omega_0 n + \phi)j} \\
        &= |A||\alpha|^n \left[ \text{cos}(\omega_0 n + \phi) \text{sin}(\omega_0 n + \phi) \right]
    \end{align}
}{Setting $A = |A|e^{j\phi}$ and $\alpha = |\alpha|e^{j\omega_0}$.}

\break

\myeq{
    \begin{align}
        x[n] &= |A||\alpha|^n e^{(\omega_1 n + \phi)j} \\
        &= |A||\alpha|^n e^{((2\pi + \omega_0)n + \phi)j} \\
        &= |A||\alpha|^n e^{(\omega_0 n + \phi)j} e^{2\pi nj} \\
        &= |A||\alpha|^n e^{(\omega_0 n + \phi)j}1
    \end{align}
    }{It could be noted that the effective frequency range in discrete systems is $2\pi$ because $n$ is an integer. As an example, here we set $\omega_1 = 2\pi + \omega_0$.}

\section{Discrete time systems}

\myeq{
    \begin{equation}
        y[n] = T\{x[n]\}
    \end{equation}
}
{General representation of discrete time systems. Note that the output sequence at each value of the index $n$ may depend on input samples $x[n]$ of all $n$.}

\myeq{
    \begin{multline} \label{eq:additivity}
        T\{x_1[n] + x_2[n]\} = T\{x_1[n]\} + T\{x_2[n]\} \\
        = y_1[n] + y_2[n]
    \end{multline}
    \begin{equation} \label{eq:scaling}
        T\{ax[n]\} = aT\{x[n]\} = ay[n]
    \end{equation}
}{\textbf{Linear systems} are defined by the principle of superposition. Equation \ref{eq:additivity} follows the property of \textit{additivity}, and Equation \ref{eq:scaling} follows the property of \textit{scaling} or \textit{homogeneity}.}

\myeq{
    \begin{align}
        y[n] &= T\{\sum^{\infty}_{-\infty}{x[k]\delta[n-k]}\} \\
        &= \sum^{\infty}_{-\infty}x[k]T\{\delta[n-k]\} \\
        &= \sum^{\infty}_{-\infty}x[k]h_k[n] \label{eq:lti}\\ 
        &= \sum^{\infty}_{-\infty}x[k]h[n-k] \label{eq:lti2}
    \end{align}
}{\textbf{Linear time-invariant systems (LTI)}. Because of the time-invariant property, $h_k[n]$ in Equation \ref{eq:lti} can be used for all values of $k$.}

\myeq{
 \begin{equation}
     y[n] = x[n] \ast h[n]
 \end{equation}
}{
    As a consequence of Equation \ref{eq:lti2}, LTI systems can be represented as a convolutional operator on the input data.
}

\myeq{
    \begin{equation} \label{eq:lti_commutative}
        x[n] \ast h[n] = h[n] \ast x[n] 
    \end{equation}
    \begin{equation} \label{eq:lti_distributive}
        x[n] \ast (h_1[n] + h_2[n]) = x[n]\ast h_1 [n] + x[n] \ast h_2[n]
    \end{equation}
    \begin{equation} \label{eq:lti_associative}
        y[n] = (x[n]\ast h_1[n])\ast h_2[n] = x[n] \ast (h_1[n] \ast h_2[n])
    \end{equation}
}{Properties of LTI systems. LTI are commutative (Equation \ref{eq:lti_commutative}), distributive (Equation \ref{eq:lti_distributive}), associative (Equation \ref{eq:lti_associative}). Therefore, multiple LTI systems can be combined into one.}

\myeq{
    \begin{equation}
        h[n] = 0 , \quad n < 0
    \end{equation}
}
{Causal LTI system only have response for $n>=0$.}










\end{document}
