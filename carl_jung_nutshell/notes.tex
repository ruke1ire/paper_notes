% pdflatex notes.tex 

\documentclass[twocolumn]{article}
\usepackage{amsfonts}
\usepackage{pifont}
\usepackage{hyperref}
\usepackage{amsmath}
\usepackage{amssymb}
\usepackage[most]{tcolorbox}
\usepackage{empheq}
\usepackage{geometry}
\usepackage{amsthm}
\geometry{a4paper, margin=1cm, includefoot}

\theoremstyle{plain}
\newtheorem{definition}{Definition}[section]

\newtcbox{\mymath}[1][]{%
    nobeforeafter, math upper, tcbox raise base,
    enhanced, colframe=blue!30!black,
    colback=blue!15, boxrule=1pt,
    #1}

\begin{document}

\title{Notes on Carl Jung:\\ 
Knowledge in a Nutshell}
\author{Rom Parnichkun}

\maketitle

\section{Jung's Psychology}

\subsection{The Word Association Experiment}

\begin{itemize}
    \item \textbf{Word association experiment}: An experiment in which patients were read a series of words and asked to free associate -- respond with the first thing that came to their mind.
    \item Patient's associated words (which is conscious) did not correlate to their diseases, instead other factors such as delay in response, changes in facial expressions, spontaneous movements, and other unconsciously stimulated behaviors seemed to correspond to possibly traumatic events.
    \item This led to the discovery of \textbf{the complex}.
\end{itemize}

\subsection{The Complex and the unconscious}

\begin{itemize}
    \item The complex is a force in the psyche that was operating beneath the surface of awareness.
    \item A complex pulls our attention in directions it might otherwise not want to go.
    \item We may observe a complex in action through a disturbance in behavior. (eg. Freudian slip, exaggerated or absence of response)
    \item The unconscious is an objective reality as independent from us as the outer world.
    \item A complex (the unconscious) is like a partial personality within us.
    \item This personality can be powerful and may overtake the conscious mind.
\end{itemize}

\subsection{The Symbolic Nature of Psyche}

\begin{itemize}
    \item Freud and Jung noticed that the unconscious spoke in the language of symbol, image, analogy and metaphor.
    \item Jung believed that images and symbols in our dreams represent messages from our unconscious that can be interpreted in multiple ways.
    \item He also believed that the meaning-making process of these symbols not only requires attendance to the real context of our particular lives and history, but also involves profound inner listening.
    \item Jung saw the unconscious as the matrix out of which consciousness emerged.
\end{itemize}

\section{The Shadow}

\subsection{Persona}

\begin{itemize}
    \item Jung took as his first principle that psychological suffering is rooted in the conflict between the instinctive nature of people and the demands imposed upon them by the society in which they lived.
    \item \textbf{Persona} refers to the inner character that we use to face the world (drawn from societal expectations, cultural norms, and natural attributes.
    \item Out ego is closely related to but not equivalent to the persona.
    \item We may become too invested in our persona, and in doing so, become unable to see the other parts of whom we really are.
    \item The persona is our somewhat embellished view of ourselves, its opposite, everything we reject about ourselves, is called the \textbf{shadow}
\end{itemize}

\subsection{The Shadow and Shadow Projection}

\begin{itemize}
    \item The shadow is a specific complex within ourselves -- an inner rejected other.
    \item Shadow projection is the projection of our shadow in the external world. For example, a bully who hates the nerd because their intelligence may be his own area of weakness.
    \item Our shadow is what can lead us to fall under the spell of collective and archetypal darkness.
    \item Jung observed that complexes could affect groups of people, this manifests itself as the \textbf{collective shadow}.
\end{itemize}


\end{document}
